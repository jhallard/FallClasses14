\documentclass[a4paper,11pt]{article}
\usepackage{amsmath}
\usepackage{fancyhdr}
\usepackage{graphicx}
\usepackage{url}
\usepackage{float}
\usepackage{amsmath}
\usepackage[margin=1in]{geometry}

\newcommand{\suchthat}{\;\ifnum\currentgrouptype=16 \middle\fi|\;}
\title{CMPE 12 - Lecture 2}
\author{ \\[7in] John Allard}
\date{October 2nd, 2014 \\}

\begin{document}

\section{Overview}
Russells paradox (end of ch1) where we say that 

$S_r$= \left\{ {\verb,S | S is a set ^ S \in S,}}\right\
%{x,y,z} is \left\{ {x, y, z}\right\} .


\section{Cartesian Products}


 \verb.Two sets, A & B. \newline
 The \emph{Cartesian Product} of (A, B) = &AxB = \left\{\verb.(a,b) a. \forall \verb.A, b. \forall B}\right\} \newline
 \verb.So if A. = $\left\{4, 3, 6}\right\}$ \verb.and B. = $\left\{circ, trian}\right\}$ \newline
 Then AxB = \left\{\verb.(4, circ) (3, circ) (6, circ) (4, triang) (3, triang) (6, triang).}\right\} \newline
 \verb.If A and B happen to be finite, then. \verb.|AxB|. = \verb.|A|x|B|. \newline

\emph{n tuples} \verb:Given A_1, A_2, A_3, ..., A_n : \newline

The cartesian of these values is = \left\{(a_1,a_2,..a_n) : a_1 : A_1 etc}\right\}

\section{Intervals}

\verb.[a, b]. = \left\{ x \in R : a <= x <= b}\right\} \newline

\section{Subsets}

A is a subset of B \iff $\forall A \in B$ \newline
A is a proper subset of B $\iff$ A $\subset$ of B and A $\neq$ B \newline
The emptyset $\subset$ of every other set \newline

If A $\not\subset$ B then \exists $ a \in A$ and $a \not\in B$

Size of empty set = 0, size of set containing only the empty set = 1.

\verb.For a set A, the power set of A is. $\wp(A)$ = \left\{$S : S \subset A$}\right\}
\section


\end{document}
`
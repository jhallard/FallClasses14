\documentclass[a4paper,11pt]{article}
\usepackage{amsmath}
\usepackage{fancyhdr}
\usepackage{graphicx}
\usepackage{url}
\usepackage{float}
\usepackage{amsmath}
\usepackage{amssymb}
\usepackage[margin=1in]{geometry}

\setlength{\voffset}{-0.5in}
\setlength{\headsep}{5pt}
\newcommand{\suchthat}{\;\ifnum\currentgrouptype=16 \middle\fi|\;}

\title{ CMPE 16 Lecture 3}
\author{John Allard, id:1437547}

\begin{document}
\maketitle
\section{Set Operation}
\subsection{Intro}
\begin{itemize}
\item For sets  $A$ and $B$, the $\cup$ of $A$ and $B$ =  $\lbrace	x : x \in A or  x \in B \rbrace$  
\item The intersection of $A$ and $B$ = $\lbrace x : x\in A and x \in B$ 
\item The difference of $A$ and $B$, $A-B$ = $\lbrace x : x \in A and x \not\in B$ 
\item $\mathbb{Z}-\mathbb{N} = \lbrace x\text{ : }x \in \mathbb{Z} \text{ and } x \leq 0 \rbrace$
\end{itemize}

\subsection{Compliments}
The complement of $A$ = $\bar{A}$ = $ \lbrace x : x \in \mathbb{U} \text{ and } x \not\in A \rbrace $ \\
Notice that to perform a complement operation you need to understand what the universe is. The answer depends on wether the universe is $\mathbb{R}^2$ or $\mathbb{N}$. 
\begin{itemize}
\item $A-B$ = $A \cap \bar{B}$
\item $\bar{A \cap B}$ = $\bar{A} \cup \bar{B}$ (DeMorgans Law) 
\item 
\end{itemize}

Examples :
\begin{itemize}
\item $\sigma = \lbrace 2n+1 : n \in \mathbb{Z}$ (odd numbers)
\item $\mathbb{P} = \lbrace n \in \mathbb{Z} : n > 1 \text{, $n$ has no factor between 1 and  $n$ }\rbrace $
\item $ S = \lbrace n^2 : n \in \mathbb{N} \rbrace $ = positive squares (0 is not in there)
\item 9 is the only square that is a multiple of 9 false,  $S \cap \mathbb{N} = \lbrace 9 \rbrace $
\item $ \mathbb{N} \cap \mathbb{P} $
\item 
\end{itemize}


\section{Index Sets}
$\sum_{n}^{i=1} i = 1+2+3 \ldots n$
So for instance, we could write the natural numbers $\mathbb{N} \cup_{i=1}^{\infty} \lbrace i \rbrace $ \\
$ \mathbb{Z} = \lbrace 0 \rbrace \cup ( \cup_{i=0}^{\infty}) \lbrace-i, i\rbrace $

\end{document}
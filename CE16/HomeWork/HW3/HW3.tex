\documentclass[a4paper,11pt]{article}
\usepackage{amsmath}
\usepackage{wrapfig}
\usepackage{fancyhdr}
\usepackage{graphicx}
\usepackage{url}
\usepackage{float}
\usepackage{amsmath}
\usepackage{amssymb}
\usepackage[margin=1in]{geometry}

\setlength{\voffset}{-0.5in}
\setlength{\headsep}{5pt}
\newcommand{\suchthat}{\;\ifnum\currentgrouptype=16 \middle\fi|\;}


%===========---------================
% Author John H Allard
% HW Assignment #3
% CMPE 16 - Discrete Math
% October 12th 2014
%===========---------================


\title{ CMPE 16 Homework \#3}
\author{John Allard, id:1437547}
\date{October 12th, 2014}

\begin{document}
\maketitle

%***************************************
%*********** HomeWork Problems *********
%************** Ten Total **************

\begin{enumerate}


%*******************************
%******** Problem # 1 ************
%*******************************
\item You have six friends, Ann, Bob, Doris, Fay, Joe and Matt. One of them always tells the
truth and the other five always lie. They each make a statement as indicated below. \\
Ann says \\
Bob says \\
Doris says \\
Fay says \\
Joe says \\
Matt says \\
“Fay tells the truth.” \\
“Ann tells the truth.” \\
“Matt or Bob tells the truth.” \\
“Doris tells the truth.” \\
“Fay lies.” \\
“Joe and I lie.” \\

\textbf{Answer :}


\item You have four friends, Meg, Pat, Zoe and Tim. Two of them always tell the truth and the
other two always lie. They each make a statement as indicated below. \\
Meg says \\
Pat says \\
Tim says \\
Zoe says \\
“I tell the truth, but Tim does not.” \\
“Tim and I are different when it comes to telling the truth.” \\
“Pat or Zoe lie. ” \\
“I tell the truth, but Tim does not.” \\
Determine which two friends tell the truth using the same technique as in the previous problem. \\
\end{enumerate}

\end{document}
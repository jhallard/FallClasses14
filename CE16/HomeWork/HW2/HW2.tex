\documentclass[a4paper,11pt]{article}
\usepackage{amsmath}
\usepackage{wrapfig}
\usepackage{fancyhdr}
\usepackage{graphicx}
\usepackage{url}
\usepackage{float}
\usepackage{amsmath}
\usepackage{amssymb}
\usepackage[margin=1in]{geometry}

\setlength{\voffset}{-0.5in}
\setlength{\headsep}{5pt}
\newcommand{\suchthat}{\;\ifnum\currentgrouptype=16 \middle\fi|\;}


%===========---------================
% Author John H Allard
% HW Assignment #2
% CMPE 16 - Discrete Math
% October 12th 2014
%===========---------================


\title{ CMPE 16 Homework \# 2}
\author{John Allard, id:1437547}
\date{October 12th, 2014}

\begin{document}
\maketitle

%***************************************
%*********** HomeWork Problems *********
%************** Ten Total **************

\begin{enumerate}


%*******************************
%******** Problem # 1 ************
%*******************************
\item \emph{Give the set represented by each of the expressions below where  $A_1 = \{ \square, 2, 8, a, g \}$,
$A_2 = \{\triangle, -2, 8, a\}$, $A_3 = \{\square, 12, 7, a, g\}$, and $A_4 = \{\square,\triangle,  2, 7, a, b, g\}$. List each element in the set only once (i.e. $\{1, 2\}$ instead of $\{1, 2, 2 \}$).}

  \begin{enumerate}
  \item $A_1 \cup A_2$ \\
  \textbf{Answer :} $A_1 \cup A_2 = \{\square, 2, 8, a, g, -2, \triangle \}$
  \item $A_3 \cap A_4$ \\
  \textbf{Answer :} $A_3 \cap A_4 = \{\square, a, 7, g \}$
  \item $A_4 - A_1$ \\
  \textbf{Answer :} $A_4 - A_1 = \{ \triangle, 7, b \}$
  \item $A_1 - A_4$ \\
  \textbf{Answer :} $A_1 - A_4 = \{ 8 \}$
  \item $\bigcup_{i=1}^{4} A_i$ \\
  \textbf{Answer :} $\bigcup_{i=1}^{4} A_i = \{ \square, 2, 8, a, g, \triangle, -2, 12, b  \}$
  \item $\bigcap_{i=1}^{4} A_i$ \\
  \textbf{Answer :} $\bigcap_{i=1}^{4} A_i = \{ a  \}$
  \end{enumerate}


%*******************************
%******** Problem # 2 ************
%*******************************

\item \emph{For each of the sets below fill in the corresponding regions of a general Venn diagram for
3 sets. (The Venn diagram should have 3 sets in each case.)}
 \textbf{Answers on last page.}
  \begin{enumerate}
  \item $A \cup \overline{B} \cup C$ 
  \item $C - ( A \cap B)$
  \item $\overline{(B - C) \cup A} $
  \end{enumerate}


%*******************************
%******** Problem # 3************
%*******************************

\item \emph{Write $ \bigcup_{i\in \mathbb{Z}} (i, i+1)$ as the difference of two well known sets. Here $(i, i+1)$ is the open interval of the real line with endpoints $i$ and $i + 1$. (That is, $(i, i + 1) = {x \in \mathbb{R} : i < x < i + 1}$).} \\ \\
\textbf{Answer :} The set $Y = \bigcup_{i\in \mathbb{Z}} (i, i+1)$ can be written as the difference of $\mathbb{R}$ and $\mathbb{N}$ ; $Y = \mathbb{R}-\mathbb{N} $. This is because $Y$ includes all numbers in $\mathbb{R}$ up-to but not including the actual integer values in $\mathbb{Z}$


-%*******************************
%******** Problem # 4************
%*******************************

\item Using only the symbols 4, Z, S, P, W, $\emptyset, \subseteq, \in, \cup, \cap,$ -, =, \{, \}, ),(, and $\neq$, express the following
statements
  
  \begin{enumerate}
  	\item 4 is pale and shy \\
  	\textbf{Answer :} $4 \in (P \cap S)$
  	\item All worried integers are pale. \\
  	\textbf{Answer :} $W \subseteq P$
  	\item Every integer is shy, worried, or pale. \\
  	\textbf{Answer :} $\mathbb{Z} \subseteq (P \cup S \cup W)$
  	\item There are worried integers that are not shy. \\
  	\textbf{Answer :} $(W - S) \neq \emptyset$
  \end{enumerate}


%*******************************
%******** Problem # 5************
%*******************************

\item Let P and Q be the statements \\ P I eat garlic. \\ Q I go to the dentist. \\
Rewrite each of the statements below using P and Q and logical connectives $(\neg, \wedge, \vee, \implies)$.

  \begin{enumerate}
  \item I don't eat Garlic \\
  \textbf{Answer :} $ \neg P$
  \item I don't go to the Dentist, but I eat garlic. \\
  \textbf{Answer :} $ \neg Q \wedge P$
  \item I eat garlic or I don't go to the dentist. \\
  \textbf{Answer :} $P \vee \neg Q$
  \item Whenever I go to the dentist, I don't eat garlic. \\
  \textbf{Answer :} $ Q \implies \neg P$
  \end{enumerate}

%*******************************
%******** Problem # 6************
%*******************************
\item Let P, Q and R be the statements \\ P I use plastic bags. \\ Q I use paper bags. \\ R I help the environment.

  \begin{enumerate}
  \item $ \neg P$ \\
  \textbf{Answer :} I do not use plastic bags
  \item $P \wedge Q$ \\
  \textbf{Answer :} I use plastic bags and I use paper bags
  \item $Q \implies \neg R$ \\
  If I use paper bags, then I am not helping the environment.
  \item $\neg (P \implies R)$ \\
  It is not the case that if I use plastic bags I will help the environment.
  \end{enumerate}

%*******************************
%******** Problem # 7 ************
%*******************************
\item Use a truth table to determine the values of each of the logical expressions below. Both of
your truth tables should have at least 3 intermediate columns.
  \begin{enumerate}
  \item $\neg (P \vee Q) \wedge \neg Q$
  \item $ (P \vee Q) \wedge (Q \vee R) \wedge \neg (P \wedge R)$
  \end{enumerate}


%*******************************
%******** Problem # 8 ************
%*******************************
\item Convert each of the following statements into the form “If P then Q” without changing
their meanings. (Some of these statements might not be True and that’s okay.)
  \begin{enumerate}
  \item For a number to be prime, it must be greater than 1. \\
  \textbf{Answer :} If $x$ is prime then $x > 1$
  \item Whenever you bring an umbrella, it rains. \\
  \textbf{Answer :} If I bring an umbrella, then it will rain.
  \item A number is prime only if it is not even. \\
  \textbf{Answer :} If a number is not even, then it is prime.  
  \item Either $n > a$ or $m > a$ if $n+m > 2a$ \\
  \textbf{Answer :} If $n+m > 2a$, then either $n > a \vee m > a$
  \end{enumerate}


%*******************************
%******** Problem # 9 ************
%*******************************
\item As discussed in class, given a finite set $S$ of size $n$ and an ordering $s_1, s_2, . . . , s_n$ of the $n$
elements in $S$, we can represent the subsets of $s$ using bit vectors of length $n$ $(\{0, 1 \}^n)$. For a subset $A \subseteq S$, the corresponding bit vector $b(A) = (b_1, b_2, . . . , b_n)$ where $b_i = 1 $ if $ s_i \in A$ and $b_i = 0$ if $ s_i  \not\in A$.
Let $S$ be the elements from the four sets in Problem 1 ordered as $\square, \triangle, -2, 2, 7, 8, 12, a, b, g$.
  \begin{enumerate}
  \item Give the bit vector corresponding to $\emptyset$ \\
  \textbf{Answer :} $ \{ 0, 0, 0, 0, 0, 0, 0, 0, 0, 0 \}$
  \item  Give the bit vector corresponding to the subset $A_1$ in Problem 1 \\
  \textbf{Answer :} $ \{ 1, 0, 0, 1, 0, 1, 0, 1, 0, 1 \}$
  \item  Give the bit vector corresponding to the subset $A_2$ in Problem 1 \\
  \textbf{Answer :} $ \{ 0, 1, 1, 0, 0, 1, 0, 1, 0, 0 \}$
  \item  Give the bit vector corresponding to the subset $A_3$ in Problem 1 \\
  \textbf{Answer :} $ \{ 1, 0, 0, 0, 1, 0, 1, 1, 0, 1\}$
  \item   Give the bit vector corresponding to the subset $A_4$ in Problem 1 \\
  \textbf{Answer :} $ \{ 1, 1, 0, 1, 1, 0, 0, 1, 1, 1\}$
  \item Give the bit vector corresponding to $S$ \\
  \textbf{Answer :} $ \{ 1, 1, 1, 1, 1, 1, 1, 1, 1, 1 \}$



  \end{enumerate}



%*******************************
%******** Problem # 10 ************
%*******************************
\item Given the bit vectors $\textbf{b}(B) = (b_1, b_2, . . . , b_n)$ and $\textbf{b}(D) = (d_1, d_2, . . . , d_n)$ representing two
subsets $B$ and $D$. In each case below explain how you would calculate the required bit vector (in
general), and then apply your method to obtain the result for $B = A_1$ and $D = A_4$ from Problem 1.
  \begin{enumerate}
  \item $\textbf{b}( B \bigcup D)$ \\
  \textbf{Answer :} In general, if I want to find the union of two bit vectors representing subsets of the same set, I would simply perform a bitwise OR operation. The OR operation would put a 1 at any index where either $B$ or $D$ have 1's, which is the definition of union. \\ In the specific case of $U = S$, $B = A_1$, and $D = A_4$ ...
  \[ \textbf{b}(A_1 \cup A_4) = \{ x_1, x_2, ..., x_n : x_i = (B_i\text{ OR }D_i)\}\]
  \[ \textbf{b}(A_1 \cup A_4) = \{ 1, 1, 0, 1, 1, 1, 0, 1, 1, 1 \}\]

  \item $ \textbf{b}( B \bigcap D) $ \\
  \textbf{Answer :} In general, if I want to find the intersection of two bit vectors representing subsets of the same set, I would simply perform a bitwise AND operation.\\ In the specific case of $U = S$, $B = A_1$, and $D = A_4$ ...
  \[ \textbf{b}(A_1 \cap A_4) = \{ x_1, x_2, ..., x_n : x_i = (B_i\text{ AND }D_i)\}\]
  \[ \textbf{b}(A_1 \cap A_4) = \{ 1, 0, 0, 1, 0, 0, 0, 1, 0, 1 \}\]
  
  \item $ \textbf{b}( B - D) $ \\
  \textbf{Answer :} To find the difference between the two subsets $B$ and $D$, I would need to only take members which are in B but also not in D. More formally,
  \[ \textbf{b}(B - D) = \{ x_1, x_2, ..., x_n : x_i = (B_i\text{ AND }(\neg D_i))\}\]
  \[ \textbf{b}(A_1 - A_4) = \{ 0, 0, 0, 0, 0, 1, 0, 0, 0, 0 \}\]

  \item $ \textbf{b}( \overline{B}) $ \\
  \textbf{Answer :} For sets, taking the compliment of a subset just means taking everything in the Universe that is not in that subset.  In other words
  \[ \textbf{b}(\overline{B}) = \{ x_1, x_2, ..., x_n : x_i = (\neg B_i)\}\]
  \end{enumerate}

\end{enumerate}

  \begin{wrapfigure}{r}{1.0\textwidth}
     \includegraphics[width=4.4in]{2Avenn}
   \caption{Venn Diagram for Problem 2a}
   \label{fig:tutorial}
  \end{wrapfigure} 

    \begin{wrapfigure}{r}{1.0\textwidth}
     \includegraphics[width=4.4in]{2Bvenn}
   \caption{Venn Diagram for Problem 2b}
   \label{fig:tutorial}
  \end{wrapfigure} 

    \begin{wrapfigure}{r}{1.0\textwidth}
     \includegraphics[width=4.4in]{2Cvenn}
   \caption{Venn Diagram for Problem 2c}
   \label{fig:tutorial}
  \end{wrapfigure} 



\end{document}
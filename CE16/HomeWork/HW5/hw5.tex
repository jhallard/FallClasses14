\documentclass[a4paper,11pt]{article}
\usepackage{amsmath}
\usepackage{wrapfig}
\usepackage{fancyhdr}
\usepackage{graphicx}
\usepackage{url}
\usepackage{float}
\usepackage{amsmath}
\usepackage{amssymb}
\usepackage[margin=1in]{geometry}

\setlength{\voffset}{-0.5in}
\setlength{\headsep}{5pt}
\newcommand{\suchthat}{\;\ifnum\currentgrouptype=16 \middle\fi|\;}
\newcommand{\answer}{\textbf{Answer : }}


%===========---------================
% Author John H Allard
% HW Assignment #4
% CMPE 16 - Discrete Math
% October 27th 2014
%===========---------================


\title{ CMPE 16 Homework \#5}
\author{John Allard}
\date{November 4th, 2014}

\begin{document}
\maketitle

%***************************************
%*********** HomeWork Problems *********
%************** Three Total ************
\begin{enumerate}


%********************************
%*********** Problem #1 *********
%********************************
\item Prove the following theorum \\
 \[ \text{For } x \in \mathbb{Z}, \text{ if $x^3-1$ is even, then $x$ is odd}\]

 \answer \\
 \begin{itemize}
 \item This will be a proof by contrapositive.
 \item Let $\mathbb{E}$ and $\mathbb{O}$ represent the even and odd integers for this proof. 
 \item $x^2-1 \in \mathbb{E} \implies x \in \mathbb{O}$ is logically equivilent to $x \in \mathbb{E} \implies x^3-1 \in \mathbb{O}$. 
 \item I will prove the later to will prove the former.
 \end{itemize}

 \begin{tabular}{l | c}
 Given some general $x \in \mathbb{E}$ & $\exists a \in \mathbb{Z} : x = 2a$ \\
 $x^3 = (2a)^3 = 8a^3$ & Cubing and substituting \\
 $8a^3-1 = 2(4a^3)-1$  & Factor out a two \\
 $\forall a \in \mathbb{Z}$ : $2(4a^3)-1 \in \mathbb{O}$ & Shows this will always be odd \\
 \end{tabular}

 Thus if $x$ is even, $x^3-1$ will always be odd, which is logically equivalent to the the fact that if $x^3-1$ is even, then $x$ must be odd.




%********************************
%*********** Problem #2 *********
%********************************
 \item Prove the following theorum \\
 \[ \text{If $x^2$ is a prime number, then $x$ is not an integer}\]

 \answer 
 
 \begin{enumerate}
 \item This will be proof by contrapositive.
 \item $x^2 \in \mathbb{P} \implies x \not\in \mathbb{Z}$ is the same as $x \in \mathbb{Z} \implies x^2 \not\in \mathbb{P}$
 \item I will directly prove the later which will prove the former.
 \item If $x \in \mathbb{Z}, then x^2 \in \mathbb{Z}$
 \item $x^2 = x*x$ which is the product of two integers that are not $1$ and $x^2$.
 \item This means that $x^2 \not\in \mathbb{P}$
 \end{enumerate}



 \item EXTRA CREDIT - Prove the following theorum

 \[ \text{If $a$, $b$, and $c$ are integers that satisfy $a^2 + b^2 + c^2$, then either $a$ or $b$ is even.} \]

 Two cases, first if $c$ is odd, second if $c$ is even.

 \begin{enumerate}
 \item $c$ is odd. If $c$ is odd, then $c^2 = (2d+1)^2 = 4d^2+4d+1$
 \item Thus $a^2 + b^2 = 4d^2+4d+1$
\end{enumerate}

\end{document}
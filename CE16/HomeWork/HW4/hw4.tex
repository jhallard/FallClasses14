\documentclass[a4paper,11pt]{article}
\usepackage{amsmath}
\usepackage{wrapfig}
\usepackage{fancyhdr}
\usepackage{graphicx}
\usepackage{url}
\usepackage{float}
\usepackage{amsmath}
\usepackage{amssymb}
\usepackage[margin=1in]{geometry}

\setlength{\voffset}{-0.5in}
\setlength{\headsep}{5pt}
\newcommand{\suchthat}{\;\ifnum\currentgrouptype=16 \middle\fi|\;}
\newcommand{\answer}{\textbf{Answer : }}


%===========---------================
% Author John H Allard
% HW Assignment #4
% CMPE 16 - Discrete Math
% October 27th 2014
%===========---------================


\title{ CMPE 16 Homework \#4}
\author{John Allard, id:1437547}
\date{October 27th, 2014}

\begin{document}
\maketitle

%***************************************
%*********** HomeWork Problems *********
%************** Ten Total **************
\begin{enumerate}


%********************************
%*********** Problem #1 *********
%********************************
\item Bridge is a card game which uses a standard deck of 52 cards. All 52 cards are dealt
to four players, so each player receives a 13-card hand. The order of the cards within a hand is not
important.
  \begin{enumerate}
  \item How many 13-card hands have exactly 9-spades? \\
  \[ {13 \choose 9} \times {39 \choose 4} = \frac{13!}{9!4!} \times \frac{39!}{35!4!} = 58809465\]

  \item How many 13-card hands have exactly 2 aces, 2 kings and 4 queens? \\
  \[ {4 \choose 2} \times {4 \choose 2} \times {4 \choose 4} = \frac{4!}{2!2!} \times \frac{4!}{2!2!} = 6^2 = 36 \]

  \item How many 13-card hands have exactly 7 face cards? \\
  \[ {12 \choose 7} \times {40 \choose 6} = \frac{12!}{5!7!} \times \frac{40!}{35!5!} = 792 \times \frac{40*39*38*37*36}{5!} = 521,142336  \]

  \item How many 13-card hands have at least 4 spades and exactly 5 hearts? \\
  \[ {13 \choose 4} \times {13 \choose 5} \times {35 \choose 4} \footnote{52 cards total, minus 13 hearts because we can't have any more of those, minus 4 spades because we can have more than 4 spades.} = \frac{13!}{9!4!} \times \frac{13!}{8!5!} \times \frac{35!}{31!4!} = 3.81819338\times10^{10} \]

  \item How many 13-card hands have 4 aces or 4 kings? \\
  \[ A = {4 \choose 4} \times {38 \choose 9} = 163,011,640 \qquad B = {4 \choose 4} \times {38 \choose 9} = 163,011,640 \]
  \[ |A \cap B| = {4 \choose 4} \times {4 \choose 4} \times {34 \choose 5} = \frac{34!}{29!}{5!} = 278,256\]
  \[ |A \cup B| = |A| + |B| - |A \cap B| = (163,011,640)^2 - 278,256 = 325,745,024 \]

  \item How many 13-card hands have cards from at most two suits? \\
  So first you have to choose which 2 suits of the four available you want to select 13 cards from. Then you can select any 13 cards from these two suits, including 13 cards that all belong to one suit.

  \[ {4 \choose 2} \times {26 \choose 13} = 6 \times \frac{26!}{13!^2} = 62,403,600\]

  \end{enumerate}



%********************************
%*********** Problem #2 *********
%********************************
\item In this problem you will calculate the number of UCSC ID numbers that meet certain
criteria, yet again. But this time, to make things simpler we will allow the first digit to be 0. So UCSC
ID numbers have 7 digits and each digit can be 0 through 9. Be careful! There are still subtleties
lurking here. 
  \begin{enumerate}
  \item How many UCSC ID numbers have exactly three 4’s ? \\
  \[ {7 \choose 3} \times {1 \choose 1}^{3} \times {9 \choose 1}^4 = 229,635\]

  \item How many UCSC ID numbers have exactly two odd digits?
  \[ {7 \choose 2} \times {4 \choose 1}^2 \times {5 \choose 1}^5 = \frac{7!}{5!2!} \times 16 \times 3125 = 1,050,000 \]

  \item How many UCSC ID numbers have at least 5 even digits?
  Choose which 5 digits must be even, select one of 5 for each of those 5 digits. For the last digits, select 1 of 10 numbers each.
  \[ {7 \choose 5} \times {5 \choose 1}^5 \times {10 \choose 1}^2 = \frac{7!}{2!5!} \times 3125 \times 100 = 6,562,500\]

  \item How many UCSC ID numbers have exactly two 5’s or exactly three 8’s?
  \[ A = \text{ID's with two 2's} = {7 \choose 2}\times{1 \choose 1}^2 \times {9 \choose 1}^5 =1,240,029 \]
  \[ B = \text{ID's with three 8's} = {7\choose3}\times{1\choose1}^3\times{9\choose1}^4 = 229,635\]

  \item How many UCSC ID numbers do not have 3 consecutive 6’s? \\
  I will find the inverse and subtract that from $10^7$ (\# of possible ID's) \\
  There are 15 ways to place (at least) 3 consecutive items in 7 consecutive slots (no proof, I just counted by hand).
  \[ 15*{1\choose1}^3* \]

  \item How many UCSC ID numbers with no repeated digits and have digits in alphabetical order (i.e.
8, 5, 4, 9,1, 7, 6, 3, 2, 0)?

  \end{enumerate}


%********************************
%*********** Problem #3 *********
%********************************
\item In each case below give the coefficient of the specified term 
  \begin{enumerate}
  \item What is the coefficient of $x^4y^3$ in $(x+y)^7$ \\
  \answer ${7 \choose 3} = \frac{7!}{3!4!} = 35$

  \item What is the coefficient of $x^7y^4$ in $(x+y)^{11}$ \\
  \answer ${11 \choose 4} = \frac{11!}{7!4!} = 330$

  \item What is the coefficient of $x^4$ in $(x+2)^{10}$ \\
  \answer ${10 \choose 6} \times 2^6 = 13,440$

  \item What is the coefficient of $x^3$ in $(2x+1)^{13}$ \\
  \answer ${13 \choose 10}\times 2^3 = 2,288$
  \end{enumerate}



%********************************
%*********** Problem #4 *********
%********************************
\item In this problem you will prove the following statement in two ways 
\[ \forall i \in \mathbb{N}, \text{ } \forall j \in \mathbb{N}, i{i-1 \choose j-1} = j{i \choose j} \]
  \begin{enumerate}
  \item By using the formula for ${n \choose k}$ \\
  Left Hand Side :
  \[ i{i-1 \choose j-1} = i\frac{(i-1)!}{i-1-(j-1))!(j-1)!} \]
  \[ i{i-1 \choose j-1} = \frac{i!}{(i-j)!(j-1)!} \]
  Right Hand Side :
  \[ j{i\choose j} = j\frac{i!}{(i-j)!j!} \]
  \[ j{i\choose j} = \frac{i!}{(i-j)!(j-1)!} \]
  Thus the two sides are equal.

  \item Using a combinatorial argument
  \end{enumerate}


%********************************
%*********** Problem #5 *********
%********************************
\item Provide direct proofs for each of the following statements. \\
  \begin{enumerate}

  \item if $n$ is an odd integer, the $n^3$ is an odd integer. \\
  \begin{tabular}{l | r}
  $n = 2x+1, x \in \mathbb{Z}$ & Definition of odd integer \\
  $n^3 = (2x+1)^3 = 8x^3+12x^2+6x+1$ & Just expanding \\
  $8x^3+12x^2+6x+1 = 2(4x^3+6x^2+3x)+1$ & Factored out a 2 \\
  $x \in \mathbb{Z} \implies (4x^3+6x^2+3x) \in \mathbb{Z}$ & Integers closed under add. and mult \\
  $n^3 = 2(4x^3+6x^2+3x)+1 = 2y+1 : y \in \mathbb{Z}$ & $n^3$ also satisfies def. of odd number.
  \end{tabular}
  Thus if $n$ is an odd integer, $n^3$ is also an odd integer.

  \item If $n$ is an odd integer, then 4 divides $n^2-1$ \\
  If $a | b$, then $\frac{b}{a} = k, k \in \mathbb{Z}$ thus $b = ka$
  \[ a = n^2-1, b = 4, \exists! q\in \mathbb{Z} : a = qb\]
  \begin{tabular}{l | r}
  $n = 2x+1, x \in \mathbb{Z}$ & Definition of odd integer \\
  $n^2-1 = (2x+1)^2-1 = 4x^2+4x$ & Substitution and Simplification \\
  $4x^2+4x = 4(x^2+x)$ & Simplifying \\
  $x \in \mathbb{Z} \implies (x^2+x) \in \mathbb{Z}$ & Integers closed under add. and mult \\
  $n^2-1 = (x^2+x)4 = q4$ ;  $ q \in \mathbb{Z}$ & Completing the proof 
  \end{tabular}

  The above shows that we can always find $q,r \in \mathbb{Z}$ such that $4 | n^2-1$ satisfies the division theorum.

  % This one seems hard
  \item


  \item for any integers $a$, $b$, and $c$, if $a | b$ and $a | (b+c)$ then $a | c$
  \[a | b \implies b = qa \quad : q \in \mathbb{Z} \qquad a | (b+c) \implies (b+c) = ta \quad :  t \in \mathbb{Z}\]
  To finish the proof, we need to show $c = ya, y \in \mathbb{Z}$ \\
  \begin{tabular}{l | r}
  $b = qa$ & Restating the facts \\
  $b+c = ta$ : $qa+c = ta$ & Substituting \\
  $c = ta-qa$ & Balancing sides \\
  $c = (t-q)a : t,q \in \mathbb{Z}$ & Factoring \\
  $c = ya, y \in \mathbb{Z}$ & subtraction of integers produces an integer \\
  \end{tabular} \\
  QED
  \end{enumerate}





\end{enumerate}

\end{document}
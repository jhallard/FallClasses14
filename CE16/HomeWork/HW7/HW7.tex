\documentclass[a4paper,11pt]{article}
\usepackage{amsmath}
\usepackage{wrapfig}
\usepackage{fancyhdr}
\usepackage{graphicx}
\usepackage{url}
\usepackage{float}
\usepackage{amsmath}
\usepackage{amssymb}
\usepackage[margin=1in]{geometry}

\setlength{\voffset}{-0.5in}
\setlength{\headsep}{5pt}
\newcommand{\suchthat}{\;\ifnum\currentgrouptype=16 \middle\fi|\;}
\newcommand{\answer}{\textbf{Answer : }}


%===========---------================
% Author John H Allard
% HW Assignment #6
% CMPE 16 - Discrete Math
% November 12th 2014
%===========---------================


% \title{ CMPE 16 Homework \#7}
% \author{John Allard}
% \date{November 11th, 2014}

\begin{document}
% \begin{titlepage}
   \vspace*{\stretch{1.0}}
   \begin{center}
      \Large\textbf{CMPE 16 Homework \#7}\\
      \large\texttt{John Allard} \\
      \small\texttt{November 18th, 2014}
   \end{center}
   \vspace*{\stretch{2.0}}
% \end{titlepage}
% \maketitle

%***************************************
%*********** HomeWork Problems *********
%************** Nine Total *************
\begin{enumerate} 


%********************************
%*********** Problem #1 *********
%********************************
\item Prove by induction that
$$ \forall n \in \mathbb{N} : \sum_{i = 1}^{n} i^3 = \frac{n^2(n+1)^2}{4}$$ \\
\textbf{ Base Case : } $n = 1$,  $ \sum_{i = 1}^{1} i^3 = 1^3 = \frac{1^2(1+1)^2}{4} = \frac{4}{4} = 1$ \\[.1in]
\textbf{ Inductive Hypothesis : } $n \in \mathbb{N}$,  $ \sum_{i = 1}^{n} i^3 = \frac{(n)^2(n+1)^2}{4}$ \\[.1in]
\textbf{ Inductive Conclusion : } $n \in \mathbb{N}$,  $ \sum_{i = 1}^{n+1} i^3 = \frac{(n+1)^2(n+2)^2}{4}$ \\[.1in]

\begin{tabular}{l | r}
$ \sum_{i = 1}^{n+1} i^3 = 1^3+2^3+3^3 \ldots + n^3 + (n+1)^3 $ & Defintion of a sum \\[.07in]
$ \sum_{i = 1}^{n+1} i^3 = \sum_{i = 1}^{n} i^3 + (n+1)^3 $     & Subsituting in the previous case \\[.07in]
$ \sum_{i = 1}^{n+1} i^3 = \frac{(n)^2(n+1)^2}{4} + (n+1)^3 $   & We know what the sum of cubes from $1$ to $n$ is. \\[.07in]
$ \sum_{i = 1}^{n+1} i^3 = (n+1)^2(\frac{n^2}{4} + (n+1)) $     & Factoring out a $(n+1)^2$ \\[.07in]
$ \sum_{i = 1}^{n+1} i^3 = (n+1)^2(\frac{n^2 + 4n + 4}{4}) $    & Put it all over one denominator \\[.07in]
$ \sum_{i = 1}^{n+1} i^3 = (n+1)^2 \frac{(n+2)^2}{4} $          & $n^2+4n+4 = (n+2)^2$ \\[.07in]
\end{tabular}

That which was to be shown has been thus shown.




\item Prove by induction that
$$ \forall k \in \mathbb{N} : \sum_{k = 1}^{n} \frac{k}{2^k} = 2 - \frac{n+2}{2^n}$$ \\
\textbf{ Base Case : } $n = 1$,  $  \sum_{k = 1}^{1} \frac{k}{2^k} = \frac{1}{2}\text{, } 2 - \frac{1+2}{2^1} = 2 - \frac{3}{2} = \frac{1}{2} $ \\[.1in]
\textbf{ Inductive Hypothesis : } $n \in \mathbb{N}$,  $ \sum_{k = 1}^{n} \frac{k}{2^k} =  2 - \frac{n+2}{2^{n}}$ \\[.1in]
\textbf{ Inductive Conclusion : } $n \in \mathbb{N}$,  $ \sum_{k = 1}^{n+1} \frac{k}{2^k} =  2 - \frac{n+3}{2^{n+1}}$ \\[.1in]

\begin{tabular}{l | r}
$ \sum_{k = 1}^{n+1} \frac{k}{2^k}=\frac{1}{2}+\frac{2}{4}+ \ldots + \frac{n}{2^n}+\frac{n+1}{2^{n+1}} $  & Defintion of our sum \\[.1in]
$ \sum_{k = 1}^{n+1} \frac{k}{2^k} = \sum_{k = 1}^{n} \frac{k}{2^k} + \frac{n+1}{2^{n+1}} $     & Subsituting summation expression for first $n$ terms \\[.1in]
$ \sum_{k = 1}^{n+1} \frac{k}{2^k} = 2 - \frac{n+2}{2^n} + \frac{n+1}{2^{n+1}} $     			& Summation expressioni s known  \\[.1in]
$ \sum_{k = 1}^{n+1} \frac{k}{2^k} = 2 - 2^{-n}(\frac{n+2}{1} - \frac{n+1}{2}) $     & Factor out $2^n$ \\[.1in]
$ \sum_{k = 1}^{n+1} \frac{k}{2^k} = 2 - 2^{-n}(\frac{2n+4 - n+1}{2}) $              & Common denominator \\[.1in]
$ \sum_{k = 1}^{n+1} \frac{k}{2^k} = 2 - 2^{-n}(\frac{n+3}{2}) $                     & Combine like terms \\[.1in]
$ \sum_{k = 1}^{n+1} \frac{k}{2^k} = 2 - \frac{n+3}{2^{n+1}} $                       & Distribute leading term\\[.1in]
\end{tabular}

That which was to be shown has been thus shown.


\item Prove by induction that
$$ \forall n \in \mathbb{N} : (n \geq 2) \implies = ((\sqrt{2}^n) \leq n!)$$ \\
\textbf{ Base Case : } $n = 2$,  $ \sqrt{2}^2 \leq 3!$ : $2 \leq 6$ \\[.1in]
\textbf{ Inductive Hypothesis : } $n \geq 2 \in \mathbb{N}$,  $ \sqrt{2}^{n} \leq (n)!$ \\[.1in]
\textbf{ Inductive Conclusion : } $n \geq 2 \in \mathbb{N}$,  $ \sqrt{2}^{n+1} \leq (n+1)!$ \\[.1in]
\begin{tabular}{l | r}
$\sqrt{2}^n \leq n!$                & Our original assumption \\
$n \geq 2$ : $\sqrt{2} < (n+1)$    &   $\sqrt{2} < 2 < (n+1)$  \\
$\sqrt{2^n}\sqrt{2} \leq (n+1)n!$   & Combining the two above equations. \footnotemark\footnotetext{If $a < b$ and $c < d$ but $a, b, c, d > 0$ then $ac < bd$}  \\
$\sqrt{2^{n+1}} \leq (n+1)! $       & Simplifying the above equation \\
\end{tabular}

Thus the inductive conclusion has been shown.



\item Prove by induction that
$$ \forall n \in \mathbb{N} : 2n^3+4n \text{ is a multiple of }3$$ \\
\textbf{ Base Case : } $n = 1$,  $ 2+4 = 6 = 2*3$ \\[.1in]
\textbf{ Inductive Hypothesis : } $n, k \in \mathbb{N}$,  $ 2n^3+4n = k*3$ \\[.1in]
\textbf{ Inductive Conclusion : } $n, j \in \mathbb{N}$,  $ 2(n+1)^3+4(n+1) = j*3$ \\[.1in]
\begin{tabular}{l | r}
$ 2(n+1)^3+4(n+1) = 2(n^3 + 3n^2 + 3n + 1) + 4n + 4$    & Distributing terms \\
$ 2(n+1)^3+4(n+1) = 2n^3 + 6n^2 + 6n + 2 + 4n + 4$      & Distributing ..  \\
$ 2(n+1)^3+4(n+1) = 2n^3 + 4n + (6n^2 + 6n + 6)$        & Grouping terms   \\
$ 2(n+1)^3+4(n+1) = k*3 + (6n^2 + 6n + 6)$              & See inductive hypthesis   \\
$ 2(n+1)^3+4(n+1) = k*3 + 3(2n^2 + 2n + 2)$             & Factor out a 3 \\
$ 2(n+1)^3+4(n+1) = 3k + 3j = 3r$                       & $k, j, r \in \mathbb{N}$   \\
\end{tabular}

The Inductive conclusion has been shown.

\item Rancher Pat is planning to raise ducks. Based on extensive research Pat estimates
that every year there will be at least 1 new baby duck for every 4 ducks in his ranch, and
that he will lose no more than 10 ducks each year. To be exact there will be $\lceil$D/4$\rceil$ new baby
ducks next year if there are D ducks in the current year. If Pat starts with 60 ducks in year
0, prove by induction that in year n Pat will have at least
$$ 20(\frac{5}{4})^n + 40 \text{ Ducks }s $$ \\
\textbf{ Base Case : } $n = 0$, $ \sum_{i=0}^{0} 60(\frac{5}{4})^0 = 60$, $20(\frac{5}{4})^0 + 40 = 60 $ \\[.1in]
\textbf{ Inductive Hypothesis : } $n \in \mathbb{N}$,  $\sum_{i=0}^{n} 60(\frac{5}{4})^n = 20(\frac{5}{4})^n + 40$ \\[.1in]
\textbf{ Inductive Conclusion : } $n \in \mathbb{N}$,  $\sum_{i=0}^{n+1} 60(\frac{5}{4})^{n+1} = 20(\frac{5}{4})^{n+1} + 40$ \\[.1in]
% \begin{tabular}{l | r}
% . \\
% \end{tabular}

\item In this problem you will prove Fermat’s Little Theorem in a different manner than we did in
class

\begin{enumerate}
\item Begin by first showing that for any prime number $p$ and integers $a$ and $b$
$$ (a+b)^p \equiv a^p + b^p \text{ mod } p$$

\item Prove by induction on $a$ that for any natural number $a$
$$ a^p \equiv a \text{ mod } p$$
\end{enumerate}


\item Prove by induction that for $n \in \mathbb{N}$ (where $F_n$ is the $n^{th}$ Fib. number)
$$ F_1^2 + F_2^2 + \ldots + F_n^2 = F_nF_{n+1}$$
\textbf{ Base Case : } $n = 0$, $ F_0 = 1$; $F_1 = 1$; $F_0F_1 = 1$ \\[.1in]
\textbf{ Inductive Hypothesis : } $n \in \mathbb{N}$,  $\sum_{i=0}^{n} F_i^2 = F_nF_{n+1}$ \\[.1in]
\textbf{ Inductive Conclusion : } $n \in \mathbb{N}$,  $\sum_{i=0}^{n+1} F_i^2 = F_{n+1}F_{n+2}$ \\[.1in]
\begin{tabular}{l | r}
$\sum_{i=0}^{n+1} F_i^2 =  F_1^2 + F_2^2 + \ldots + F_n^2 + F_{n+1}^2 $    & Defintion of our summation \\
$\sum_{i=0}^{n+1} F_i^2 =  \sum_{i=0}^{n} F_i^2  + F_{n+1}^2 $	   		   & Substitute summation for first $n$ terms  \\
$\sum_{i=0}^{n+1} F_i^2 =  F_nF_{n+1}  + F_{n+1}^2 $	   		   		   & Substitute Inductive Hypothesis  \\
$\sum_{i=0}^{n+1} F_i^2 =  F_nF_{n+1}  + F_{n+1}F_{n+1} $	   		   	   & Simplifying-ish  \\
$\sum_{i=0}^{n+1} F_i^2 =  F_{n+1}(F_{n}  + F_{n+1}) $	   		   		   & Factoring  \\
$\sum_{i=0}^{n+1} F_i^2 =  F_{n+1}F_{n+2} $	   		   		  			   & $F_{n+2} = F_n + F_{n+1}$  \\
\end{tabular}
Thus the inductive conclusion has been shown.






\end{enumerate}


\end{document}
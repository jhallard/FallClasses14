\documentclass[a4paper,11pt]{article}
\usepackage{amsmath}
\usepackage{wrapfig}
\usepackage{fancyhdr}
\usepackage{graphicx}
\usepackage{url}
\usepackage{float}
\usepackage{amsmath}
\usepackage{amssymb}
\usepackage[margin=1in]{geometry}

%\setlength{\voffset}{-0.5in}
%\setlength{\headsep}{5pt}
\newcommand{\suchthat}{\;\ifnum\currentgrouptype=16 \middle\fi|\;}
\newcommand{\answer}{\textbf{Answer : }}


%===========---------================
% Author John H Allard
% CMPE 12, Lab #2 Write-up
% October 9th, 2014
%===========---------================


\title{ CMPE 12 Homework \#2 \\[7 in]}
\author{John Allard \\ Lab Section \#2}
\date{October 29th, 2014}

\begin{document}
\maketitle
\newpage

%==========================================
%==========================================
%====== Begin Problems, 15 Total ==========
%==========================================
%==========================================

\begin{enumerate}

% 4.5
\item Refer to the table supplied in the textbook for the next few questions.
  
  \begin{enumerate}
  \item What binary value does location 3 contain? What about location 6? \\
  \answer Location 3 contains the binary value \small{0000 0000 0000 0000}, location 6 contains \small{000 0110 1101 1001}.

  \item Binary values can be interpreted in different ways, interpret the values in the given table as instructed so by the probnlems below.
    \begin{enumerate}
    \item Interpret location 0 and location 1 in two's compliment. \\
    \answer Location 0 contains 7,747, location 1 contains -4,059.
    \item Interpret location 4 as an ASCII value. \\
    \answer `e'.
    \item Interpret locations 6 and 7 and an IEEE floating point number. (7 being the high bits, 6 being the low). \\
    \answer \small{0 00001101 10110011111111011010011} = 8.20007582580913043462315872541E-35 
    \item Interpret locations 1 and 0 and unsigned integers.
    \answer Location 0 contains 7,474, and location 1 contains 61,477.
    \end{enumerate}

  \item Interpret the data in location 0 as an instruction. \\
  \answer The opcode \small{0001} corresponds to the ADD operation. This operation has a \small{0} in the 5th bit place, which means it is operating on two source registers. The operation is $R7= R1+R3$.

  \item Interpret the data in location 5 as an address, then state the data that is at that address. \\
  \answer Location 5 contains the address 6. Location 6 contains the value \texttt{0xfed3}.
  \end{enumerate} 

% 4.7
\item Suppose a 32-bit opcode takes the format of 
\begin{tabular}{|c|c|c|c|}
\hline OPCODE & SR & DR & IMM \\
\hline
\end{tabular}

If there are 60 opcodes and 32 registers, what is the range of values that can be represented by the intermediate (IMM)? \\
\answer To represent 60 opcodes, you would need 6 bits (this would give you 4 extra). To address 32 registers, you would need exactly 5 bits. Thus to state an opcode and 2 registers, you would need to use 16 bits. This would leave 16 bits for the IMM. $2^{16} = 65,536$. Thus any addressed within 65,536 of the current position can be accessed.

% 4.8
\item Using the same 32-bit instruction format as above, except there needs to be 225 opcodes and a 120 registers to address.
  \begin{enumerate}
  \item Minimum number of bits to represent an opcode? \answer 8 bits.
  \item Min. number of bits to represent the Destination register? \answer 7 bits. 
  \item What is the maximum number of unused bits in the instruction encoding? \answer 3 unused bits.
  \end{enumerate}

% 4.14
\item Describe the execution of the JMP instruction if R3 contains \texttt{x369C}. \\
\answer This instruction will be stored at x36A2. The IR is loading with the JMP instruction, and the PC is incremented to x36A3. Then, when the jump command is processed, the PC is made to point to x369C, when the RET command is found, it will jump back to x36A3 and continue execution.

% 5.4
\item If we have a memory consisting of 256 locations, and each location contains 16 bits.
  \begin{enumerate}
  \item How many bits are required for the address? \answer 8 bits
  \item If we use the PC-relative addressing mode, and want to allow control transfer between instructions 20 locations away, how many bits of a branch instruction are needed to specify the pc-relative offset? \answer 5 bits, would allow us to jump up to 32 spaces away.
  \item If a control instruction is in location 3, what is the PC-relative offset of address 10? Assume that the control transfer instructions work the same wat as in the LC-3.
  \end{enumerate}

% 5.14
\item Fill in the missing two sequences below to perform an OR operation. (Lines 1,3 are given, 2 and 4 need to be filled by the student). The operation is being performed on registers 1 and 2 and will be stored in register 3.
\begin{enumerate}
\item \texttt{1001 100 001 1111111} - NOT R1 into R4.
\item \texttt{1001 000 010 1111111} - NOT R2 into R5.
\item \texttt{0101 110 100 000 101} - AND R4 and R5 into R6
\item \texttt{1001 011 110 1111111} - NOT R6 into R3
\end{enumerate}

The OR is accomplished using DeMorgans laws, which states that $\neg(\neg A \wedge \neg B) = A \vee B$

% 5.15
\item State the contents of the R1, R2, R3, and R4 registers after the program starting at \texttt{x3100} halts.

% 6.16
\item An LC-3 program is located in memory locations \texttt{x3000} to \texttt{x3006}. It starts executing at the former, if we keep track of all values loaded into the MAR as the program executed, we will get a sequence that starts as follows (See book). Your job is to fill in each blank space with either a 0 or 1 as appropriate.

\begin{tabular}{| c || c | c | c | c | c | c | c | c | c | c | c | c | c | c | c | c | c | c | c | c | c | c | c |} \hline 
x3000 & 0 & 0 & 1 & 0 & 0 & 0 & 0 & 0 &0 & 0 & 0 & 0 & 0 & 1 & 0 & 0 \\ \hline
x3001 & 0 & 0 & 0 & 1 & 0 & 0 & 0 & 0 & 0 & 0 & 1 & 0 & 0 & 0 & 0 & 1 \\ \hline
x3002 & 1 & 0 & 1 & 1 & 0 & 0 & 0 & 0 & 0 & 0 & 0 & 0 & 0 & 0 & 1 & 1 \\ \hline
x3003 & 0 & 0 & 0 & 0 & 0 & 0 & 0 & 0 & 0 & 0 & 0 & 0 & 0 & 0 & 0 & 0 \\ \hline
x3004 & 1 & 1 & 1 & 1 & 0 & 0 & 0 & 0 & 0 & 0 & 1 & 0 & 0 & 1 & 0 & 1 \\ \hline
x3005 & 0 & 0 & 0 & 0 & 0 & 0 & 0 & 0 & 0 & 0 & 0 & 1 & 1 & 0 & 0 & 0 \\ \hline
x3006 & 0 & 1 & 0 & 0 & 0 & 0 & 0 & 0 & 0 & 0 & 0 & 0 & 0 & 0 & 0 & 1 \\ \hline
\end{tabular} \\
It starts at x3000, gets a LD instruction from x3005 into R0, then it adds 1 to this value, then it stores this value. 

% 6.18
\item The LC3 has no divide instruction. A programmer needing to divide two numbers would have to write a routine to handle it. Show the systematic decomp. of the process of dividing two positive integers. Write an LC-3 machine language program starting at location x3000 which divides the number in memory location x4000 by the number in memory location x4001 and stores the quotient at x5000 and the remainder at x5001.

% 7.5
\item For the program listed on page 191 of the book (Chapter 7, problem 5), what does the program do and what is the the value contained in the RESULT (x0000) address after the program halts?


% 7.10
\item The program fragment shown on page 192 of the book has an error in it. Identify the error and state how to fix it.

%8.8
\item Write a program that checks the initial value in memory location x4000 to see if it is a valid ASCII code. If it is, print the char. Else, exit the program.

%8.16
\item What does the LC-3 program on page 217 (Chapter 8, problem 16) do?

%9.1
\item 

%9.2
\item

%9.4
\item
\end{enumerate}

\end{document}